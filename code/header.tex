%\usepackage{fontspec} %加這個就可以設定字體
\usepackage{xeCJK} %讓中英文字體分開設置
%\usepackage{indentfirst} %加上這個第一段才會空兩格。\noindent加在該段前面,可以取消該段的空兩個。
%\usepackage{float} %use the 'float' package

%cmd指令找字體>>fc-list :lang=zh
\setmainfont{Times New Roman} %設定英文字體,初始為Computer Modern,但是要額外下載。
%\defaultCJKfontfeatures{AutoFakeBold=6, AutoFakeSlant=.4} %以後不用再設定粗斜
\setCJKmainfont{Microsoft JhengHei}
\setCJKsansfont{Microsoft JhengHei}%xeCJK裡面\secCJKsansfont需要顯示定義
\setCJKmonofont{Microsoft JhengHei}%設定中文為系統上的字型,而英文不去更動,使用原TeX字型

\usepackage{xcolor}
%\setlength{\parindent}{2em} %會讓每段空兩個,但是第一段除外。
\definecolor{shadecolor}{rgb}{.9, .9, .9} %程式區塊的背景顏色

%解決 R markdown 的heading 4問題。
\makeatletter %'@' is now a normal "letter" for TeX
\renewcommand\paragraph{\@startsection{paragraph}{4}{\z@}%
        {-2.5ex\@plus -1ex \@minus -.25ex}%
        {1.25ex \@plus .25ex}%
        {\normalfont\normalsize\bfseries}}
\makeatother %'@' is restored as a "non-letter" character for TeX
\setcounter{secnumdepth}{4} % how many sectioning levels to assign numbers to




%
%\usepackage{fontspec} %加這個就可以設定字體
%\usepackage{xeCJK} %讓中英文字體分開設置
%\setCJKmainfont{Microsoft YaHei} %設定中文為系統上的字型,而英文不去更動,使用原TeX字型
%\XeTeXlinebreaklocale "zh" %這兩行一定要加,中文才能自動換行。%表示用中文的斷行
%\XeTeXlinebreakskip = 0pt plus 1pt %這兩行一定要加,中文才能自動換行。%預留調整的空間


